%!TEX root = ../thesis.tex
% ******************************* Thesis Appendix B ********************************

\chapter{DirectCompute Vector Addition} \label{appendix:DirectComputeVecAdd}

\lstset{language=C++,
                keywordstyle=\color{blue},
                stringstyle=\color{BurntOrange},
                commentstyle=\color{OliveGreen},
                basicstyle=\footnotesize,
                numbers=left,
                stepnumber=1,
                tabsize=4,
                showstringspaces=false
}

\section{Vector Addition Compute Shader (.hlsl)}
\begin{lstlisting}
// Vector addition compute shader
RWBuffer<int>		Buffer0		: register(u0);
RWBuffer<int>		Buffer1		: register(u1);
RWBuffer<int>		BufferOut   : register(u2);

[numthreads(1, 1, 1)]
void CSMain(uint3 DTid : SV_DispatchThreadID)
{
	BufferOut[DTid.x] = Buffer0[DTid.x] + Buffer1[DTid.x];
}

\end{lstlisting}





\section{Vector Addition main (.cpp) }

\begin{lstlisting}

#include <stdio.h>
#include <d3d11.h>
#include <d3dcompiler.h>
#include <iostream>

#include "dxerr.h"


#pragma comment(lib,"d3d11.lib")
#pragma comment(lib,"d3dcompiler.lib")

#ifndef SAFE_RELEASE
#define SAFE_RELEASE(p)      { if (p) { (p)->Release(); (p)=nullptr; } }
#endif
#ifndef CHECK_ERR
#define CHECK_ERR(x)    { hr = (x); if( FAILED(hr) ) { DXTraceW( __FILEW__, __LINE__, hr, L#x, true ); exit(1); } }
#endif


#define NUM_ELEMENTS	1024


HRESULT CreateComputeDevice(ID3D11Device** deviceOut, ID3D11DeviceContext** contextOut, bool bForceRef){
	*deviceOut = nullptr;
	*contextOut = nullptr;

	HRESULT hr = S_OK;

	// We will only call methods of Direct3D 11 interfaces from a single thread.
	UINT flags = D3D11_CREATE_DEVICE_SINGLETHREADED;
	D3D_FEATURE_LEVEL featureLevelOut;
	static const D3D_FEATURE_LEVEL flvl[] = { D3D_FEATURE_LEVEL_11_0, D3D_FEATURE_LEVEL_10_1, D3D_FEATURE_LEVEL_10_0 };

	bool bNeedRefDevice = false;
	if (!bForceRef)
	{
		hr = D3D11CreateDevice(nullptr,     // Use default graphics card
			D3D_DRIVER_TYPE_HARDWARE,		// Try to create a hardware accelerated device
			nullptr,                        // Do not use external software rasterizer module
			flags,							// Device creation flags
			flvl,
			sizeof(flvl) / sizeof(D3D_FEATURE_LEVEL),
			D3D11_SDK_VERSION,				// SDK version
			deviceOut,						// Device out
			&featureLevelOut,               // Actual feature level created
			contextOut);					// Context out

		if (SUCCEEDED(hr))
		{
			// A hardware accelerated device has been created, so check for Compute Shader support

			// If we have a device >= D3D_FEATURE_LEVEL_11_0 created, full CS5.0 support is guaranteed, no need for further checks
			if (featureLevelOut < D3D_FEATURE_LEVEL_11_0)
			{
				// Otherwise, we need further check whether this device support CS4.x (Compute on 10)
				D3D11_FEATURE_DATA_D3D10_X_HARDWARE_OPTIONS hwopts;
				(*deviceOut)->CheckFeatureSupport(D3D11_FEATURE_D3D10_X_HARDWARE_OPTIONS, &hwopts, sizeof(hwopts));
				if (!hwopts.ComputeShaders_Plus_RawAndStructuredBuffers_Via_Shader_4_x)
				{
					bNeedRefDevice = true;
					printf("No hardware Compute Shader capable device found, trying to create ref device.\n");
				}
			}
		}
	}

	if (bForceRef || FAILED(hr) || bNeedRefDevice)
	{
		// Either because of failure on creating a hardware device or hardware lacking CS capability, we create a ref device here
		SAFE_RELEASE(*deviceOut);
		SAFE_RELEASE(*contextOut);

		hr = D3D11CreateDevice(nullptr,				// Use default graphics card
			D3D_DRIVER_TYPE_REFERENCE,				// Try to create a hardware accelerated device
			nullptr,								// Do not use external software rasterizer module
			flags,									// Device creation flags
			flvl,
			sizeof(flvl) / sizeof(D3D_FEATURE_LEVEL),
			D3D11_SDK_VERSION,						// SDK version
			deviceOut,								// Device out
			&featureLevelOut,						// Actual feature level created
			contextOut);							// Context out
		if (FAILED(hr))
		{
			printf("Reference rasterizer device create failure\n");
			return hr;
		}
	}

	return hr;
}

HRESULT CreateComputeShader(LPCWSTR pSrcFile, LPCSTR pFunctionName, ID3D11Device* pDevice, ID3D11ComputeShader** ppShaderOut)
{
	HRESULT hr = S_OK;

	DWORD dwShaderFlags = D3DCOMPILE_ENABLE_STRICTNESS;

	const D3D_SHADER_MACRO defines[] =
	{
		"USE_STRUCTURED_BUFFERS", "1",
		nullptr, nullptr
	};


	// We generally prefer to use the higher CS shader profile when possible as CS 5.0 is better performance on 11-class hardware
	LPCSTR pProfile = (pDevice->GetFeatureLevel() >= D3D_FEATURE_LEVEL_11_0) ? "cs_5_0" : "cs_4_0";


	ID3DBlob* pErrorBlob = nullptr;
	ID3DBlob* pBlob = nullptr;

#if D3D_COMPILER_VERSION >= 46
	hr = D3DCompileFromFile(pSrcFile, defines, D3D_COMPILE_STANDARD_FILE_INCLUDE, pFunctionName, pProfile, dwShaderFlags, 0, &pBlob, &pErrorBlob);
#else
	hr = D3DX11CompileFromFile(pSrcFile, defines, nullptr, pFunctionName, pProfile, dwShaderFlags, 0, nullptr, &pBlob, &pErrorBlob, nullptr);
#endif

	if (FAILED(hr))
	{
		if (pErrorBlob){
			std::cout << (char*)pErrorBlob->GetBufferPointer() << std::endl;
		}


		SAFE_RELEASE(pErrorBlob);
		SAFE_RELEASE(pBlob);

		return hr;
	}



	hr = pDevice->CreateComputeShader(pBlob->GetBufferPointer(), pBlob->GetBufferSize(), nullptr, ppShaderOut);

	SAFE_RELEASE(pErrorBlob);
	SAFE_RELEASE(pBlob);


	return hr;
}

HRESULT CreateRawBuffer(ID3D11Device* pDevice, UINT uElementSize, UINT uCount, void* pInitData, ID3D11Buffer** ppBufOut){
	*ppBufOut = nullptr;

	D3D11_BUFFER_DESC desc;
	ZeroMemory(&desc, sizeof(desc));

	desc.BindFlags = D3D11_BIND_UNORDERED_ACCESS | D3D11_BIND_SHADER_RESOURCE;
	desc.ByteWidth = uElementSize * uCount;
	desc.MiscFlags = D3D11_RESOURCE_MISC_BUFFER_ALLOW_RAW_VIEWS;
	desc.StructureByteStride = uElementSize;

	if (pInitData)
	{
		D3D11_SUBRESOURCE_DATA InitData;
		InitData.pSysMem = pInitData;
		return pDevice->CreateBuffer(&desc, &InitData, ppBufOut);
	}
	else
		return pDevice->CreateBuffer(&desc, nullptr, ppBufOut);
}

HRESULT CreateBufferUAV(ID3D11Device* pDevice, ID3D11Buffer* pBuffer, ID3D11UnorderedAccessView** ppUAVOut)
{
	D3D11_BUFFER_DESC descBuf;
	ZeroMemory(&descBuf, sizeof(descBuf));
	pBuffer->GetDesc(&descBuf);

	D3D11_UNORDERED_ACCESS_VIEW_DESC desc;
	ZeroMemory(&desc, sizeof(desc));
	desc.ViewDimension = D3D11_UAV_DIMENSION_BUFFER;
	desc.Buffer.FirstElement = 0;

	if (descBuf.MiscFlags & D3D11_RESOURCE_MISC_BUFFER_ALLOW_RAW_VIEWS)
	{
		// This is a Raw Buffer

		desc.Format = DXGI_FORMAT_R32_TYPELESS; // Format must be DXGI_FORMAT_R32_TYPELESS, when creating Raw Unordered Access View
		desc.Buffer.Flags = D3D11_BUFFER_UAV_FLAG_RAW;
		desc.Buffer.NumElements = descBuf.ByteWidth / 4;
	}
	else
		if (descBuf.MiscFlags & D3D11_RESOURCE_MISC_BUFFER_STRUCTURED)
		{
		// This is a Structured Buffer

		desc.Format = DXGI_FORMAT_UNKNOWN;      // Format must be must be DXGI_FORMAT_UNKNOWN, when creating a View of a Structured Buffer
		desc.Buffer.NumElements = descBuf.ByteWidth / descBuf.StructureByteStride;
		}
		else
		{
			return E_INVALIDARG;
		}

	return pDevice->CreateUnorderedAccessView(pBuffer, &desc, ppUAVOut);
}

ID3D11Buffer* CreateAndCopyToDebugBuf(ID3D11Device* pDevice, ID3D11DeviceContext* pd3dImmediateContext, ID3D11Buffer* pBuffer)
{
	ID3D11Buffer* debugbuf = nullptr;

	D3D11_BUFFER_DESC desc;
	ZeroMemory(&desc, sizeof(desc));
	pBuffer->GetDesc(&desc);
	desc.CPUAccessFlags = D3D11_CPU_ACCESS_READ;
	desc.Usage = D3D11_USAGE_STAGING;
	desc.BindFlags = 0;
	desc.MiscFlags = 0;



	if (SUCCEEDED(pDevice->CreateBuffer(&desc, nullptr, &debugbuf)))
	{
		pd3dImmediateContext->CopyResource(debugbuf, pBuffer);
	}


	return debugbuf;
}

int main()
{
	ID3D11Device* device = nullptr;
	ID3D11DeviceContext* context = nullptr;
	ID3D11ComputeShader* CS = nullptr;

	if (FAILED(CreateComputeDevice(&device, &context, false))) exit(1);
	if (FAILED(CreateComputeShader(L"../VectorAddCS.hlsl", "CSMain", device, &CS))) exit(1);

	HRESULT hr = S_OK;

	// Buffers
	ID3D11Buffer* b_inA = nullptr;
	ID3D11Buffer* b_inB = nullptr;
	ID3D11Buffer* b_out = nullptr;

	// Access views
	ID3D11UnorderedAccessView*  b_inA_UAV  = nullptr;
	ID3D11UnorderedAccessView*  b_inB_UAV  = nullptr;
	ID3D11UnorderedAccessView*  b_out_UAV = nullptr;

	// Setup data
	int *i_inA = new int[NUM_ELEMENTS];
	int *i_inB = new int[NUM_ELEMENTS];
	int *i_out = new int[NUM_ELEMENTS];

	for (int i = 0; i < NUM_ELEMENTS; i++){
		i_inA[i] = i;
		i_inB[i] = 2 * i;
	}

	// Create buffers
	CHECK_ERR(CreateRawBuffer(device, sizeof(int), NUM_ELEMENTS, &i_inA[0], &b_inA));
	CHECK_ERR(CreateRawBuffer(device, sizeof(int), NUM_ELEMENTS, &i_inB[0], &b_inB));
	CHECK_ERR(CreateRawBuffer(device, sizeof(int), NUM_ELEMENTS, nullptr,   &b_out));


	// Create access views
	CHECK_ERR(CreateBufferUAV(device, b_inA, &b_inA_UAV));
	CHECK_ERR(CreateBufferUAV(device, b_inB, &b_inB_UAV));
	CHECK_ERR(CreateBufferUAV(device, b_out, &b_out_UAV));

	// Launch CS
	{
		context->CSSetShader(CS, nullptr, 0);
		ID3D11UnorderedAccessView* aRViews[3] = { b_inA_UAV, b_inB_UAV, b_out_UAV };
		context->CSSetUnorderedAccessViews(0, 3, aRViews, nullptr);
		context->Dispatch(NUM_ELEMENTS, 1, 1);

		// Unmap resources
		ID3D11UnorderedAccessView* aRViewsNullptr[3] = { nullptr, nullptr, nullptr };
		context->CSSetUnorderedAccessViews(0, 3, aRViewsNullptr, nullptr);
	}

	// Retrieve results
	{
		// Retrieve positions
		ID3D11Buffer* resDebugbuf = CreateAndCopyToDebugBuf(device, context, b_out);
		D3D11_MAPPED_SUBRESOURCE mappedRes;

		context->Map(resDebugbuf, 0, D3D11_MAP_READ, 0, &mappedRes);
		memcpy(i_out, (int*)mappedRes.pData, NUM_ELEMENTS*sizeof(int));
		context->Unmap(resDebugbuf, 0);

	}

	for (size_t i = 0; i < NUM_ELEMENTS; i++) {
		if (i_inA[i] + i_inB[i] != i_out[i]) return 1;
	}


	std::cout << "Sucess!" << std::endl;

    return 0;
}


\end{lstlisting}