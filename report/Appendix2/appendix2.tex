%!TEX root = ../thesis.tex
% ******************************* Thesis Appendix B ********************************

\chapter{OpenCL Vector Addition} \label{appendix:OpenCLVecAdd}
\lstset{language=C++,
                keywordstyle=\color{blue},
                stringstyle=\color{BurntOrange},
                commentstyle=\color{OliveGreen},
                basicstyle=\footnotesize,
                numbers=left,
                stepnumber=1,
                tabsize=4,
                showstringspaces=false
}

\begin{lstlisting}
#include <CL/cl.hpp>
#include <iostream>
#include <vector>

const int N_elements = 1024;

// Kernel source, is usually stored in a seperate .cl file
std::string src = 
"__kernel void vectorAddition("
"	__global read_only int* vector1, \n"
"	__global read_only int* vector2, \n"
"	__global write_only int* vector3)\n"
"{\n"
"	int indx = get_global_id(0);\n"
"	vector3[indx] = vector1[indx] + vector2[indx];\n"
"}\n";

int main()
{

	// Get list of available platforms
	std::vector<cl::Platform> platforms;
	cl::Platform::get(&platforms);

	// Get list of devices
	std::vector<cl::Device> devices;
	platforms[0].getDevices(CL_DEVICE_TYPE_ALL, &devices);

	// Create a context from the devices
	cl::Context context(devices);

	// Compile kernel
	cl::Program program(context, cl::Program::Sources(1, std::make_pair(src.c_str(), src.length() + 1)));
	cl_int err = program.build(devices, "-cl-std=CL1.2");

	// Input data
	int* vector1 = new int[N_elements];
	int* vector2 = new int[N_elements];

	for (size_t i = 0; i < N_elements; i++){
		vector1[i] = i;
		vector2[i] = 2 * i;
	}

	// Create buffers from input data
	cl::Buffer vec1Buff(context, CL_MEM_READ_ONLY, sizeof(int) * N_elements, vector1);
	cl::Buffer vec2Buff(context, CL_MEM_READ_ONLY, sizeof(int) * N_elements, vector2);

	int* vector3 = new int[N_elements];
	cl::Buffer vec3Buff(context, CL_MEM_WRITE_ONLY, sizeof(int) * N_elements, vector3);

	// Pass arguments to the vector addition kerel
	cl::Kernel kernel(program, "vectorAddition", &err);
	kernel.setArg(0, vec1Buff);
	kernel.setArg(1, vec2Buff);
	kernel.setArg(2, vec3Buff);
	
	// Create command queue and copy data to the device
	cl::CommandQueue queue(context, devices[0]);
	queue.enqueueWriteBuffer(vec1Buff, CL_TRUE, 0, sizeof(int) * N_elements, vector1);
	queue.enqueueWriteBuffer(vec2Buff, CL_TRUE, 0, sizeof(int) * N_elements, vector2);

	// Launch kernel
	queue.enqueueNDRangeKernel(kernel, cl::NullRange, cl::NDRange(N_elements), cl::NDRange(1024));

	// Read back result
	queue.enqueueReadBuffer(vec3Buff, CL_TRUE, 0, sizeof(int) * N_elements, vector3);

	// Assert that the result is correct
	for (size_t i = 0; i < N_elements; i++){
		if (vector1[i] + vector2[i] != vector3[i])
			return 1;
	}

	std::cout << "Sucess!" << std::endl;

	delete[] vector1;
	delete[] vector2;
	delete[] vector3;

	return 0;
}
\end{lstlisting}